\documentclass[12pt]{memoir}
\usepackage{setspace}

\usepackage{geometry}
 \geometry{
 a4paper,
 total={155mm,257mm},
 left=30mm,
 top=20mm
  }

\linespread{1.6}
\title{Rain}
\date{}

\setlength{\parindent}{0cm}
\begin{document}
\maketitle
\noindent
--I'm telling you, I don't have a religion. Why is that strange?\\
--I shouldn't have said that. Unusual, yes, like someone with a doublename. Paul Paul. Not strange.\\
--You know I'm half Syrian Christian. My family, I mean. I don't have a religion. My grandfather's name is Abraham Abraham. Or was---he died in 2002. It's funny how you deny a human the right to the present when they stop being, isn't it? I \emph{had} a grandfather whose name \emph{was} Abraham Abraham.\\
--The Human who Stopped Being. To be no more. That shouldn't be synonymous with death. A lot of the people I know are no more, most of the time anyway.\\
--It was my parents' idea.\\
--Abraham Abraham?\\
--No, don't be silly. Religion. They've left the decision to me, or so they say. They're quite irreligious themselves; it wouldn't surprise me if they were atheists. Of course, I wasn't brought up an atheist either. I was brought up to think critically; told to question what I was told; there is no absolute truth; seek, and inconsistency you shall find, and so on. How's your shake?\\
--Not so bad. Cold.\\
--May I have a sip? I've finished mine. What is it, a blushberry frapp\'{e} ?\\
--Yes. Go ahead. \\
\hrule 
\vspace{\baselineskip}
--There! You see? You called me a pig. Distinctly. A sweaty pig. With my snuffling snout in your glass, slurping \emph{your} blushberry frapp\'{e} through \emph{your} straw. But I can't help sweating. It's so hot, and your blushberry is so cold.\\
--Come on. You would never say that.\\
--You're right. Supposing I talked about the weather instead--- \\
\hrule
\vspace{\baselineskip}
look, it's starting to drizzle. Soon, the rain'll be so loud that we'll hardly be able to hear ourselves talk about it.\\
--I love when it's like that. When it rains so hard it drowns your voice out. Sometimes, especially at night with the lights off, I open the window just a little and sit on the window sill and let the breeze blow the spray in my face. And drink tea and listen to the silence. It's one of the silences you can hear.\\
\hrule 
\vspace{\baselineskip}
--That's beautiful. I should have told you how beautiful that was\\
--I wish you had.\\
--even if you did say you loved it when my voice was drowned out.\\
--I did. \\
\hrule 
\vspace{\baselineskip}
Let's leave now, I'm getting a little sick of this place.\\
--Don't you want to finish your milkshake?\\
--Not really. Not enough to want to stay.\\
--It's pouring. Where do you want to go?\\
--Let's take a cab and go sit by the sea. Could I have the bill, please?\\
--Do you have an umbrella? I never thought of it before, but whenever I read about rain it's cold and stinging. Or icy. The only rain I've felt is like this, though, lukewarm and torrential.\\
--Like a good piss. When I was younger I would hold it in for as long as I could because it would feel so good when I went. My mum did it too when she was little. She calls it the unbearable lightness of peeing. No I don't have an umbrella. Do you have one big enough for us both?\\
--I think so. Split it fifty-fifty?\\
--Yes. Sixty each.\\
--Imagine if the rain were cold and stinging. You wouldn't be able to sit at your windowsill. Here, I'll hold the umbrella. Imagine if it were hailing.\\
--I wish it would hail a taxi.\\
--What? I didn't hear you.\\
--I said I wish it would hail a Taxi! Taxi! Get in.\\
--You get in first. I'm holding the umbrella.\\
--You're quite handsome when you're lost in thought, you know.\\
--Handsome? Come on.\\
--No really. Like a tall, dark stranger. What are you thinking about?\\
\hrule 
\vspace{\baselineskip}
--I wanted to say about you but it was the truth.\\
--I hoped you would.\\
\hrule 
\vspace{\baselineskip}
--The problem with tall dark strangers is that when you get to know them they're only tall and dark. His toe was blown off by a bomb, did I ever tell you? Abraham Abraham's.\\
--No. I'd like to hear about it, though.\\
--Not an actual bomb. He wasn't communist enough for that. It was a firecracker on Diwali night. Possibly even a Lakshmi bomb; I remember those, like sticks of dynamite with her picture on them. Vividly coloured. It used to bother me as a child that people exploded a goddess in celebration of her\ldots \ yes, Abraham Abraham's toe. Well, he was wheeling his scooter home---too dangerous to ride, you see, with all the firecrackers littering the street---after the birth of his first grandchild. My cousin. Christian baby comes into the world with a bang on Hindu festival. Anyway, he was wheeling it home, probably not carefully enough, when he trod on Lakshmi-not-yet-exploded who took his toe off for it. I suppose there was a lot of blood; I was never told that part. Some un-christian swearing, certainly. Funny that it was Diwali, you know---no more ten-headed Ravana, no more ten-toed Abraham Abraham.\\
--That's a good story.\\
--It always seemed oddly auspicious to him that his first grandchild should be born on Diwali. Profanely auspicious, I should think, for a Christian. I was born on the evening of a Hindu festival myself, Janmashtami, but since I was his third, I suppose it wasn't as bad. To my grandmother---not Abraham Abraham's wife, the other one, the Hindu grandmother---it's the reason for my complexion. Born under the same moon as deep-blue-skinned Krishna.\\
--Poetic. But really, this would make a good story. You should write it down.\\
--I think I will, someday. I'm telling it to you, that'll help me remember.\\
--Do. Stories don't write themselves.\\

\end{document}